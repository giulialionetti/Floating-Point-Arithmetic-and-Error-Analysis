\documentclass[11pt]{article}
\usepackage{amsmath}
\usepackage{amssymb}
\usepackage{amsthm}
\usepackage{geometry}
\usepackage{enumitem}
\usepackage{hyperref}
\usepackage{algorithm}
\usepackage{algpseudocode}
\usepackage{booktabs}

\geometry{margin=1in}

\theoremstyle{definition}
\newtheorem{theorem}{Theorem}[section]
\newtheorem{definition}[theorem]{Definition}
\newtheorem{property}[theorem]{Property}

\title{Interval Arithmetic, Verification Methods, and Multiple Precision}
\author{}
\date{}

\begin{document}

\maketitle

\tableofcontents
\newpage

\section{Interval Arithmetic}

\subsection{Basic Interval Operations}

\begin{definition}[Interval notation]
\begin{equation}
\mathbf{x} = [\underline{x}; \overline{x}] = \{x \in \mathbb{R} : \underline{x} \leq x \leq \overline{x}\}
\end{equation}
\end{definition}

\begin{definition}[Midpoint and width]
\begin{align}
\text{mid}(\mathbf{x}) &= \frac{\overline{x} + \underline{x}}{2}\\
w(\mathbf{x}) &= \overline{x} - \underline{x}
\end{align}
\end{definition}

\subsection{Operations with Directed Rounding}

\subsubsection{Addition}
\begin{equation}
\mathbf{x} + \mathbf{y} = [\nabla(\underline{x} + \underline{y}), \Delta(\overline{x} + \overline{y})]
\end{equation}

\subsubsection{Subtraction}
\begin{equation}
\mathbf{x} - \mathbf{y} = [\nabla(\underline{x} - \overline{y}), \Delta(\overline{x} - \underline{y})]
\end{equation}

\subsubsection{Multiplication}
\begin{equation}
\mathbf{x} \times \mathbf{y} = [\min\{\underline{x}\underline{y}, \underline{x}\overline{y}, \overline{x}\underline{y}, \overline{x}\overline{y}\}, \max\{\underline{x}\underline{y}, \underline{x}\overline{y}, \overline{x}\underline{y}, \overline{x}\overline{y}\}]
\end{equation}

\subsubsection{Division}

If $0 \notin [\underline{y}, \overline{y}]$:
\begin{equation}
\mathbf{x}/\mathbf{y} = \mathbf{x} \times \frac{1}{\mathbf{y}}
\end{equation}

where
\begin{equation}
\frac{1}{\mathbf{x}} = [1/\overline{x}; 1/\underline{x}]
\end{equation}

\section{Interval Newton Method}

\begin{definition}[Interval Newton operator]
\begin{equation}
N(\tilde{x}, \mathbf{X}) := \tilde{x} - \frac{f(\tilde{x})}{f'(\mathbf{X})}
\end{equation}
\end{definition}

\begin{theorem}[Convergence conditions]
Let $\mathbf{X}$ be an interval and $\tilde{x} \in \mathbf{X}$. Assume $0 \notin f'(\mathbf{X})$.
\begin{itemize}
    \item If $N(\tilde{x}, \mathbf{X}) \subset \mathbf{X}$, then $\mathbf{X}$ contains a \textbf{unique root} of $f$
    \item If $N(\tilde{x}, \mathbf{X}) \cap \mathbf{X} = \emptyset$, then $\mathbf{X}$ contains \textbf{no roots} of $f$
\end{itemize}
\end{theorem}

\section{Fixed Point Theorem (Brouwer)}

\subsection{For Nonlinear Systems}

For nonlinear systems:
\begin{equation}
f(\mathbf{x}) = 0 \Leftrightarrow g(\mathbf{x}) = \mathbf{x}
\end{equation}

where
\begin{equation}
g(\mathbf{x}) := \mathbf{x} - Rf(\mathbf{x}) \quad \text{with } \det(R) \neq 0
\end{equation}

\begin{theorem}[Key result]
\begin{equation}
\mathbf{X} \in \mathbb{IR}^n, \quad g(\mathbf{X}) \subseteq \mathbf{X} \quad \Rightarrow \quad \exists \hat{\mathbf{x}} \in \mathbf{X}, \quad g(\hat{\mathbf{x}}) = \hat{\mathbf{x}}
\end{equation}
\end{theorem}

\subsection{Mean Value Form}

\begin{equation}
-Rf(\tilde{\mathbf{x}}) + (I - RM)\mathbf{Y} \subseteq \mathbf{Y} \quad \Rightarrow \quad g(\mathbf{X}) \subseteq \mathbf{X}
\end{equation}

where $M$ is an interval matrix containing all Jacobians in $\mathbf{X}$.

\section{Matrix Nonsingularity Test}

\begin{theorem}
\begin{equation}
|I - RA| < 1 \quad \Rightarrow \quad A \text{ is nonsingular}
\end{equation}
\end{theorem}

\noindent \textbf{Application}: Choose $R \approx A^{-1}$ and compute $|I - RA|$ with interval arithmetic.

\section{Verification of Nonlinear Systems}

\begin{theorem}[Verification theorem for nonlinear systems]
Let $f : \mathbb{R}^n \to \mathbb{R}^n$ with $f \in C^1$, $\tilde{\mathbf{x}} \in \mathbb{R}^n$, $\mathbf{X} \in \mathbb{IR}^n$ with $0 \in \mathbf{X}$.

Let $M \in \mathbb{IR}^{n \times n}$ such that:
\begin{equation}
\{\nabla f_i(\zeta) : \zeta \in \tilde{\mathbf{x}} + \mathbf{X}\} \subseteq M_{i,:}
\end{equation}

\textbf{If}:
\begin{equation}
-Rf(\tilde{\mathbf{x}}) + (I - RM)\mathbf{X} \subseteq \text{int}(\mathbf{X})
\end{equation}

\textbf{Then}:
\begin{itemize}
    \item There exists a \textbf{unique} $\hat{\mathbf{x}} \in \tilde{\mathbf{x}} + \mathbf{X}$ with $f(\hat{\mathbf{x}}) = 0$
    \item The Jacobian $J_f(\hat{\mathbf{x}})$ is nonsingular
\end{itemize}
\end{theorem}

\section{Multiple Roots Verification}

For double roots, solve the system:
\begin{equation}
G(\mathbf{x}, e) = \begin{pmatrix} f(x) - e \\ f'(x) \end{pmatrix} = 0
\end{equation}

\noindent \textbf{Jacobian}:
\begin{equation}
J_G(x, e) = \begin{pmatrix} f'(x) & -1 \\ f''(x) & 0 \end{pmatrix}
\end{equation}

This system is \textbf{well-conditioned} for double roots (avoids the ill-conditioning).

\section{Multiple Precision Arithmetic}

\subsection{Double-Double Numbers}

\begin{definition}[Double-double]
A \textbf{double-double} is a pair $(a_h, a_l)$ satisfying:
\begin{equation}
a = a_h + a_l \quad \text{and} \quad |a_l| \leq u|a_h|
\end{equation}
where $u$ is the unit roundoff of the base precision.
\end{definition}

\begin{theorem}[Error bound for double-double operations]
\begin{equation}
\text{fl}(a \odot b) = (1 + \delta)(a \odot b), \quad |\delta| \leq 4 \cdot 2^{-106}
\end{equation}
where $\odot \in \{+, \times\}$
\end{theorem}

\noindent \textbf{Precision}: Double-double in IEEE 754 double precision gives approximately \textbf{106 bits} of precision (double the 53 bits of standard double precision).

\subsection{Representation Formats}

\subsubsection{Multiprecision number}
\begin{equation}
s \cdot m \cdot \beta^e
\end{equation}

where:
\begin{itemize}
    \item $s$: sign
    \item $m$: mantissa (arbitrary length)
    \item $\beta$: base (typically 2)
    \item $e$: exponent
\end{itemize}

\subsubsection{With integers}
\begin{equation}
m = \sum_{i=0}^{n} m_i B^i
\end{equation}
where $m_i$ are machine integers and $B$ is the word size.

\subsubsection{With expansions}
\begin{equation}
x = \sum_{i=0}^{n} f_i
\end{equation}
where $f_i$ are floating-point numbers with non-overlapping mantissas.

\section{Kulisch Accumulator}

\noindent \textbf{Purpose}: Compute exact sum/dot product without rounding errors.

\subsection{Register Length}

For exact dot product (double precision):
\begin{equation}
L = k + 2e_{\max} + 2|e_{\min}| + 2n = 4288 \text{ bits}
\end{equation}

where:
\begin{itemize}
    \item $n = 53$ bits (mantissa precision)
    \item $e_{\min} = -1022$ (minimum exponent)
    \item $e_{\max} = 1023$ (maximum exponent)
    \item $k = 92$ bits (for products: $2n - 2 = 104$ rounded up)
\end{itemize}

\noindent \textbf{Key property}: All intermediate results fit exactly in the accumulator, so only one rounding occurs at the end.

\section{Error Analysis}

\subsection{Forward vs Backward Error}

\begin{equation}
\text{forward error} \approx \text{condition number} \times \text{backward error}
\end{equation}

\subsubsection{Number of Correct Digits}
\begin{equation}
\text{Number of correct digits} = -\log_{10}(\text{forward error})
\end{equation}

\noindent \textbf{Rule of thumb}:
\begin{equation}
\text{Number of correct digits} \approx -\log_{10}(u) - \log_{10}(\kappa)
\end{equation}
where $u$ is the unit roundoff and $\kappa$ is the condition number.

\subsection{Multiple Roots}

\noindent \textbf{Key insight}: Forward error is $O(u^{1/m})$

\begin{itemize}
    \item Single root ($m=1$): error $\sim u$ \checkmark\ well-conditioned
    \item Double root ($m=2$): error $\sim \sqrt{u}$ \texttimes\ ill-conditioned
    \item Triple root ($m=3$): error $\sim u^{1/3}$ \texttimes\texttimes\ very ill-conditioned
\end{itemize}

\noindent \textbf{Multiple roots are always ill-conditioned!}

\section{Key Algorithms}

\subsection{TwoSum (Error-Free Transformation)}

\begin{algorithm}[H]
\caption{TwoSum}
\begin{algorithmic}[1]
\Function{TwoSum}{$a, b$}
    \State $s \gets \text{fl}(a + b)$
    \State $z \gets \text{fl}(s - a)$
    \State $t \gets \text{fl}(b - z)$
    \State \Return $(s, t)$
\EndFunction
\end{algorithmic}
\end{algorithm}

\begin{property}
$s + t = a + b$ (exact), where $s$ is the rounded sum and $t$ is the rounding error.
\end{property}

\subsection{FastTwoSum (when $|a| \geq |b|$)}

\begin{algorithm}[H]
\caption{FastTwoSum}
\begin{algorithmic}[1]
\Function{FastTwoSum}{$a, b$}
    \State $s \gets \text{fl}(a + b)$
    \State $z \gets \text{fl}(s - a)$
    \State $t \gets \text{fl}(b - z)$
    \State \Return $(s, t)$
\EndFunction
\end{algorithmic}
\end{algorithm}

\begin{property}
Same as TwoSum but requires $|a| \geq |b|$. One fewer operation than TwoSum.
\end{property}

\noindent \textbf{Critical}: Uses \textbf{Sterbenz's lemma}: if $a/2 \leq b \leq 2a$, then $a - b$ is exact.

\subsection{TwoProduct (Error-Free Transformation)}

\begin{algorithm}[H]
\caption{TwoProduct}
\begin{algorithmic}[1]
\Function{TwoProduct}{$a, b$}
    \State $p \gets \text{fl}(a \times b)$
    \State $e \gets \text{FMA}(a, b, -p)$ \Comment{or use Dekker's algorithm}
    \State \Return $(p, e)$
\EndFunction
\end{algorithmic}
\end{algorithm}

\begin{property}
$p + e = a \times b$ (exact), where $p$ is the rounded product and $e$ is the error.
\end{property}

\section{Condition Numbers (Quick Reference)}

\subsection{For Different Problems}

\subsubsection{Summation}
\begin{equation}
\text{cond}\left(\sum p_i\right) = \frac{\sum |p_i|}{|\sum p_i|}
\end{equation}

\subsubsection{Matrix-vector product}
\begin{equation}
\text{cond}(Ax) \leq |A| |x| / |Ax|
\end{equation}

\subsubsection{Linear systems ($Ax = b$)}
\begin{equation}
\kappa(A) = |A| |A^{-1}|
\end{equation}

\subsubsection{Polynomial evaluation at $x$}
\begin{equation}
\text{cond}(p, x) = \frac{\tilde{p}(|x|)}{|p(x)|}
\end{equation}
where $\tilde{p}$ has absolute value coefficients.

\subsubsection{Polynomial root (simple root $\alpha$)}
\begin{equation}
K(p, \alpha) = \frac{\tilde{p}(|\alpha|)}{|\alpha| |p'(\alpha)|}
\end{equation}

\section{Common Error Bounds}

\subsection{Summation (Recursive)}

\begin{equation}
\left|\frac{\tilde{s} - s}{s}\right| \leq nu \cdot \text{cond}\left(\sum p_i\right)
\end{equation}

where $n$ is the number of terms and $u$ is the unit roundoff.

\subsection{Matrix Multiplication}

\subsubsection{Inner product $x^T y$}
\begin{equation}
\text{fl}(x^T y) = \sum_{i=1}^n x_i y_i (1 + \theta_i), \quad |\theta_i| \leq nu
\end{equation}

\subsubsection{Matrix-matrix $AB$}
\begin{equation}
\text{fl}(AB)_{ij} = \sum_{k=1}^n a_{ik} b_{kj} (1 + \theta_{ijk}), \quad |\theta_{ijk}| \leq nu
\end{equation}

\subsection{Linear Systems}

\subsubsection{LU factorization}
\begin{equation}
\text{fl}(LU) = A + E, \quad |E| \leq nu |A|
\end{equation}

\subsubsection{Forward error after solving}
\begin{equation}
\frac{|\tilde{x} - x|}{|x|} \lesssim \kappa(A) \cdot u
\end{equation}

\subsubsection{Iterative refinement (one iteration)}
\begin{equation}
\frac{|\tilde{x} - x|}{|x|} \lesssim \kappa(A)^2 \cdot u
\end{equation}

\section{Key Symbols and Notation}

\begin{table}[h]
\centering
\begin{tabular}{ll}
\toprule
\textbf{Symbol} & \textbf{Meaning} \\
\midrule
$\nabla$ & Rounding toward $-\infty$ (round down) \\
$\Delta$ & Rounding toward $+\infty$ (round up) \\
$\mathbb{IR}$ & Set of intervals \\
$u$ & Unit roundoff (machine epsilon) \\
& \quad fp64: $u = 2^{-53} \approx 1.11 \times 10^{-16}$ \\
& \quad fp32: $u = 2^{-24} \approx 5.96 \times 10^{-8}$ \\
& \quad fp16: $u = 2^{-11} \approx 4.88 \times 10^{-4}$ \\
$\text{int}(\mathbf{X})$ & Interior of interval $\mathbf{X}$ \\
$\kappa(A)$ & Condition number of matrix $A$ \\
$\text{fl}(x)$ & Floating-point representation of $x$ \\
$\tilde{x}$ & Computed (approximate) value \\
$\hat{x}$ & Exact value \\
\bottomrule
\end{tabular}
\end{table}

\section{Quick Tips for the Exam}

\begin{enumerate}
    \item \textbf{Always check}: Is the problem well-conditioned? ($\kappa \ll 1/u$)
    
    \item \textbf{Multiple roots}: Reformulate using the $G(x, e)$ system to avoid ill-conditioning
    
    \item \textbf{Interval Newton}: Check $0 \notin f'(\mathbf{X})$ before claiming uniqueness
    
    \item \textbf{Directed rounding}:
    \begin{itemize}
        \item Round down for lower bounds: $\nabla$
        \item Round up for upper bounds: $\Delta$
    \end{itemize}
    
    \item \textbf{Double-double}: Gives $\sim2\times$ precision ($\sim106$ bits vs 53 bits)
    
    \item \textbf{Iterative refinement}: Needs high-precision residual ($u_r$) to converge
    
    \item \textbf{Error accumulation}:
    \begin{itemize}
        \item Worst case: $O(nu)$
        \item Probabilistic: $O(\sqrt{n}u)$ with random rounding
    \end{itemize}
\end{enumerate}

\end{document}
